\sloppy

% Настройки стиля ГОСТ 7-32
% Для начала определяем, хотим мы или нет, чтобы рисунки и таблицы нумеровались в пределах раздела, или нам нужна сквозная нумерация.
\EqInChapter % формулы будут нумероваться в пределах раздела
\TableInChapter % таблицы будут нумероваться в пределах раздела
\PicInChapter % рисунки будут нумероваться в пределах раздела

% Добавляем гипертекстовое оглавление в PDF
\usepackage[
bookmarks=true, colorlinks=true, unicode=true,
urlcolor=black,linkcolor=black, anchorcolor=black,
citecolor=black, menucolor=black, filecolor=black,
]{hyperref}

\usepackage{graphicx}   % Пакет для включения рисунков

% Пакет Tikz
\usepackage{tikz}
\usetikzlibrary{arrows,positioning,shadows}

% Произвольная нумерация списков.
\usepackage{enumerate}


% Шрифты. Это работает только с XeLaTeX'ом
\usepackage{fontspec}
\usepackage{xunicode}
\usepackage{xltxtra}
\defaultfontfeatures{Mapping=tex-text, Scale=MatchLowercase}
\newfontfamily{\cyrillicfont}{Times}
\setmainfont{Times}
\setromanfont{Times}
%\usepackage{polyglossia}
%\setdefaultlanguage{russian}
%\setmainlanguage{russian}
%\setotherlanguage{english}
