%% Преамбула TeX-файла

% 1. Стиль и язык
\documentclass[utf8x, 12pt]{G7-32} % Стиль (по умолчанию будет 14pt)

% Остальные стандартные настройки убраны в preamble.inc.tex.
\sloppy

% Настройки стиля ГОСТ 7-32
% Для начала определяем, хотим мы или нет, чтобы рисунки и таблицы нумеровались в пределах раздела, или нам нужна сквозная нумерация.
\EqInChapter % формулы будут нумероваться в пределах раздела
\TableInChapter % таблицы будут нумероваться в пределах раздела
\PicInChapter % рисунки будут нумероваться в пределах раздела

% Добавляем гипертекстовое оглавление в PDF
\usepackage[
bookmarks=true, colorlinks=true, unicode=true,
urlcolor=black,linkcolor=black, anchorcolor=black,
citecolor=black, menucolor=black, filecolor=black,
]{hyperref}

\usepackage{graphicx}   % Пакет для включения рисунков

% Пакет Tikz
\usepackage{tikz}
\usetikzlibrary{arrows,positioning,shadows}

% Произвольная нумерация списков.
\usepackage{enumerate}


% Шрифты. Это работает только с XeLaTeX'ом
\usepackage{fontspec}
\usepackage{xunicode}
\usepackage{xltxtra}
\defaultfontfeatures{Mapping=tex-text, Scale=MatchLowercase}
\newfontfamily{\cyrillicfont}{Times}
\setmainfont{Times}
\setromanfont{Times}
%\usepackage{polyglossia}
%\setdefaultlanguage{russian}
%\setmainlanguage{russian}
%\setotherlanguage{english}


% Настройки листингов.
\include{example/listings.inc}

% Полезные макросы листингов.
\include{example/macros.inc}

\begin{document}

\frontmatter % выключает нумерацию ВСЕГО; здесь начинаются ненумерованные главы: реферат, введение, глоссарий, сокращения и прочее.

% Команды \breakingbeforechapters и \nonbreakingbeforechapters
% управляют разрывом страницы перед главами.
% По-умолчанию страница разрывается.

% \nobreakingbeforechapters
% \breakingbeforechapters

\include{example/00-abstract}

\tableofcontents

\Defines % Необходимые определения. Вряд ли понадобться
\begin{description}
  \item[Облачные Технологии] (англ. cloud computing), модель обеспечения повсеместного и удобного сетевого доступа по требованию к общему пулу (англ. pool) конфигурируемых вычислительных ресурсов (например, сетям передачи данных, серверам, устройствам хранения данных, приложениям и сервисам — как вместе, так и по отдельности), которые могут быть оперативно предоставлены и освобождены с минимальными эксплуатационными затратами и/или обращениями к провайдеру.
  \item[Amazon Elastic Compute Cloud] Веб-сервис, который предоставляет вычислительные мощности в облаке. Сервис входит в инфраструктуру Amazon Web Services.
  \item[Amazon Web Services] Облачная платформа от компании Amazon, в которой представлено много сервисов для предоставления различных услуг, таких как: хранение данных, аренда виртуальных серверов, предоставление вычислительных мощностей и др.
\end{description}
\Abbreviations %% Список обозначений и сокращений в тексте
\begin{description}
  \item[ОС] Операционная Система
  \item[ПО] Программное Обеспечение
  \item[API] Интерфейс программирования приложений (англ. application programming interface)
  \item[AWS] Amazon Web Services
  \item[EC2] Amazon Elastic Compute Cloud

\end{description}



\Introduction
В большинстве магазинов розничной торговли есть камеры видеонаблюдения. При помощи алгоритмов компьютерного зрения, их можно использовать для получения различных статистических параметров, которые могут помочь владельцам магазинов лучше настроить бизнес процессы. Это может быть кол-во поситетелей, длина очереди, самая популярная витрина, и т.п.

Разрабатываемое в компании Cera Marketing решение позволяет полностью автоматизировать сбор данных с камер, их обработку и получение результатов.




\mainmatter % это включает нумерацию глав и секций в документе ниже

\include{example/20-analysis}
\include{example/30-design}
\include{example/40-impl}
\chapter{Экспериментальный раздел}
\label{cha:research}

В данном разделе проводятся вычислительные эксперименты.
А на рис.~\ref{fig:spire01} показана схема мыслительного процесса автора...

\begin{figure}
  \centering
  \includegraphics[width=\textwidth]{assets/cage.jpg}
  \caption{Как страшно жить}
  \label{fig:spire01}
\end{figure}


%%% Local Variables:
%%% mode: latex
%%% TeX-master: "rpz"
%%% End:


\backmatter %% Здесь заканчивается нумерованная часть документа и начинаются ссылки и
            %% заключение

\include{example/80-conclusion}

\include{example/81-biblio}

\appendix   % Тут идут приложения

\chapter{Картинки}
\label{cha:appendix1}

\begin{figure}
\centering
\includegraphics[width=\textwidth]{assets/cage.jpg}
\caption{Картинка в приложении. Страшная и ужасная.}
\end{figure}

%%% Local Variables:
%%% mode: latex
%%% TeX-master: "rpz"
%%% End:

\chapter{Еще картинки}
\label{cha:appendix2}

\begin{figure}
\centering
\includegraphics[width=\textwidth]{assets/cage.jpg}
\caption{Еще одна картинка, ничем не лучше предыдущей. Но надо же как-то заполнить место.}
\end{figure}

%%% Local Variables:
%%% mode: latex
%%% TeX-master: "rpz"
%%% End:


\end{document}

%%% Local Variables:
%%% mode: latex
%%% TeX-master: t
%%% End:
